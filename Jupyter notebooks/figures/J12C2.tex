\documentclass[border=4pt]{standalone}
\usepackage{tikz}
\usepackage[siunitx, RPvoltages]{circuitikz}

\begin{document}

\ctikzset{bipoles/length=1.5cm}
\ctikzset{bipoles/thickness=4}
\ctikzset{label/align = smart}

\ctikzset{
 amplifiers/fill=cyan,
 sources/fill=cyan!30!white,
 csources/fill=green!30!white,
 diodes/fill=red,
 resistors/fill=violet,
 }

\begin{tabular}{ll}
\begin{circuitikz}[american, scale = 1.0, cute inductors]]
	\draw (0,0) to[I,l=$i_s$, -] (0,3);	         
	\draw (2,3) to[R, l=$R$, v=$v_R$, f=$i_R$,*-*] (2,0);
	             
	\draw (4,3) to[L, l=$L$, v=$v_L$, f=$i_L$,*-*] (4,0);
	\draw (6,3) to[C, l=$C$, v_=$v_C$, f=$i_C$] (6,0);          
  
	\draw (0,0)--(6,0);
	\draw (0,3)--(6,3);
\end{circuitikz} & 
\begin{circuitikz}[american, scale = 1.0, cute inductors]]
	\draw (0,0) to[V, l=$v_s$, -] (0,3)	         
	            to[R, l=$R$, v=$v_R$, f=$i_R$,*-*] (3,3);
	             
	\draw (3,3) to[L, l=$L$, f=$i_L$,*-*] (3,0);
	\draw (5,3) to[C, l=$C$, f=$i_C$] (5,0);          
  
	\draw (0,0)--(5,0);
	\draw (3,3)--(5,3);
		
\end{circuitikz} 
\end{tabular}


\end{document}