\documentclass[border=4pt]{standalone}
\usepackage{tikz}
\usetikzlibrary{arrows.meta,arrows}
\usepackage[siunitx, RPvoltages]{circuitikz}

\begin{document}

\ctikzset{bipoles/length=1.5cm}
\ctikzset{bipoles/thickness=4}
\ctikzset{label/align = smart}

\ctikzset{
 amplifiers/fill=cyan,
 sources/fill=cyan!30!white,
 csources/fill=green!30!white,
 diodes/fill=red,
 resistors/fill=violet,
 }


\begin{circuitikz}[american, scale = 1.0, cute inductors]]

    \draw (0,0) to[R=2<\ohm>, *-*] (-3,0)
                to[I,l=$0.25$A, *-*](-3,3)
                to[R=3<\ohm>, *-*] (0,6);
                
    \draw (-3,3)to[R=4<\ohm>, *-*] (0,3);
                 
	\draw (0,0) to[V=2<\volt>, -*] (0,3)
	            to[I,l=$0.5$A,*-*] (3,3)
	            to[R=2<\ohm>, *-*] (6,3);
	\draw (0,3) to[R=3<\ohm>, *-*] (0,6)	            
	            to[R=3<\ohm>, *-*] (3,3);           
         	            
	
	\draw (3,3) to[R=1<\ohm>, *-*] (3,0);
    
    \draw (6,3) node[anchor=west]{a};
    \draw (6,0) node[anchor=west]{b};
    
    \draw (-6,3) node[anchor=east]{c};
    \draw (-6,0) node[anchor=east]{d};
    
    \draw node[ocirc] (a) at (6,3){}
	      node[ocirc] (b) at (6,0){}
	      (a) to[open,*-*] (b);   
	      
	\draw node[ocirc] (c) at (-6,3){}
	      node[ocirc] (d) at (-6,0){}
	      (c) to[open,*-*] (d); 
	        
	\draw (0,0)--(6,0);
	\draw (-3,3) to[R=1<\ohm>, *-*](-6,3);
	\draw (-3,0)--(-6,0);	
	      	
\end{circuitikz}


\end{document}